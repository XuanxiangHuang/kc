
\section{Compiling CNF into Plain DNNF}
%
Given a CNF formula $\phi$ on variable set $\mbf{V}$, we compile
$\phi$ into a \emph{plain DNNF} by recursively partitioning $\mbf{V}$ into
variable regions and applying clause--variable decomposition at each subregion.


\subsection{Clause Splitting}
%
Let $(\mbf{X},\mbf{Y})$ be a partition of $\mbf{V}$ such that $\mbf{X} \cap \mbf{Y} = \emptyset$
and $\mbf{X} \cup \mbf{Y} = \mbf{V}$.
For an arbitrary clause $\gamma \in \phi$ defined on $\mbf{V}$, let $\gamma = \gamma(\mbf{X}) \lor \gamma(\mbf{Y})$,
where $\gamma(\mbf{X})$ and $\gamma(\mbf{Y})$ are the sub-clauses of $\gamma$
mentioning variables $\mbf{X}$ and $\mbf{Y}$, that is,
\[
\gamma(\mbf{X}) := \bigvee_{\ell \in \gamma,\; \text{var}(\ell)\in \mbf{X}}\; \ell,
\qquad
\gamma(\mbf{Y}) := \bigvee_{\ell \in \gamma,\; \text{var}(\ell)\in \mbf{Y}}\; \ell.
\]
A clause $\gamma$ is called a \emph{cross-clause} w.r.t. $(\mbf{X},\mbf{Y})$ if
it contains variables from both $\mbf{X}$ and $\mbf{Y}$.
And a CNF $\phi$ can be divided into three parts, clauses $\gamma$ that only mention $\mbf{X}$,
clauses $\gamma$ that only mention $\mbf{Y}$, and cross-clauses $\gamma_i$ that mention
both $\mbf{X}$ and $\mbf{Y}$.

\begin{definition}[Clausal $(\mbf{X},\mbf{Y})$-decomposition]
Given a CNF $\phi = \{\gamma_1,\dots,\gamma_k\}$,
the clausal $(\mbf{X},\mbf{Y})$-decomposition of a CNF $\phi$ is defined as
\[
CD(\phi,\mbf{X},\mbf{Y}) = \bigg\{\bigg(\bigcup_{i\in S} \gamma_{i}(\mbf{X}), \bigcup_{i\not\in S} \gamma_{i}(\mbf{Y}) \bigg) \bigg| S\subseteq \{1,\dots, k\} \bigg\}.
\]
Then $\phi$ can be written as follows
\begin{equation}
\label{eq:cd}
\phi
\;=\;
\bigvee_{S}
\big(
L_{S}(\mbf{X})
\;\wedge\;
R_{S}(\mbf{Y})
\big),
\end{equation}
where
\[
L_{S}(\mbf{X}) := \bigg( \bigwedge_{i\in S}\gamma_{i}(\mbf{X}) \bigg),
\qquad
R_{S}(\mbf{Y}) := \bigg( \bigwedge_{i\not\in S} \gamma_{i}(\mbf{Y}) \bigg).
\]
And inconsistent pairs $(L_{S},R_{S})$ are removed.
\end{definition}


\subsection{Recursive Decomposition}
%
We outline a simple recursive compilation algorithm.
\begin{algorithm}[H]
\caption{$\msf{Compile}(\phi, \mbf{V})$}
%
\begin{algorithmic}[1]
\Require CNF $\phi$ over variable set $\mbf{V}$
\Ensure A plain DNNF equivalent to $\phi$
\If{$\phi$ has no cross-clauses w.r.t. $(\mbf{X},\mbf{Y})$}
\State \Return a DNNF for $\phi$
\EndIf
\State Choose one partition $(\mbf{X},\mbf{Y})$ of $\mbf{V}$
\Comment{Heuristic techniques}
\State Compute $CD(\phi,\mbf{X},\mbf{Y})$
\State For each surviving pair $(L_{S},R_{S})$, recursively compile:
\Comment{Parallelizm on pairs is optional}
\[
\psi_{S} = \msf{Compile}(L_{S}, \mbf{X}) \wedge \msf{Compile}(R_{S}, \mbf{Y}).
\]
\State \Return $\bigvee_{S} \psi_{S}$.
\end{algorithmic}



\label{alg:compile}
\end{algorithm}
%
If a CNF $\phi$ does not include cross-clauses, then $\phi$ is trivially decomposable, and a DNNF
can be constructed immediately.
However, if $\phi$ contains cross-clauses, then we need to heuristically decompose it.

The algorithm then applies an $(\mbf{X},\mbf{Y})$-decomposition to split $\phi$
into $\mbf{X}$-components and $\mbf{Y}$-components according to equation~\eqref{eq:cd}, 
and recursively invokes itself to further compile each pair $(L_{S}, R_{S})$.
Note that each pair $(L_{S}, R_{S})$ can be compiled independently, which makes parallelization possible.
Moreover, for each pair $(L_{S}, R_{S})$, one can locally choose the most suitable $(\mbf{X}, \mbf{Y})$-decomposition.
Finally, the algorithm returns a disjunction over all DNNFs compiled from the pairs $(L_{S}, R_{S})$ as the resulting DNNF equivalent to $\phi$.
Several aspects of the algorithm~\ref{alg:compile} can be summarized as follows:
\begin{itemize}
\item \textbf{Diverse partitions across regions.}
Distinct subformulas (e.g., $L_{S}(\mbf{X})$, $L_{S'}(\mbf{X})$) may choose
different internal partitions, yielding a heterogeneous
hierarchy of region refinements.
\item \textbf{Parallel compilation.}
The subproblems $(L_{S},R_{S})$ produced by each
$CD(\phi,\mbf{X},\mbf{Y})$ are independent and may be compiled in
parallel.  Identical subregions can be merged.
\end{itemize}



\section{Choosing the Best $(\mbf{X},\mbf{Y})$-Decomposition}
%
In the proposed recursive compilation framework, each invocation
$\msf{Compile}(\phi,\mbf{V})$ requires selecting a partition
$(\mbf{X},\mbf{Y})$ of the current variable set~$\mbf{V}$.
The quality of this partition strongly affects circuit size:
fewer cross-clauses lead to fewer disjunctions introduced by
$CD(\phi,\mbf{X},\mbf{Y})$ and thus to smaller DNNFs.
This section outlines theoretical tools and heuristics that can guide
the choice of $(\mbf{X},\mbf{Y})$.
\xnote{
There are a plenty studies of parameters to measure complexity of compilation
~\cite{oztok2014compiling,bova2015compiling,oztok2015top,capelli2021certifying,de2025compilation}
Apparently, there exists additional parameters for bounding the complexity
of the proposed compilation algorithm, which we need to formally define.
}


